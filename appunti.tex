\documentclass[a4paper, 11pt]{article}
\usepackage[a4paper, left=3cm, bottom=4cm, right=3cm]{geometry}

\RequirePackage[english]{babel}
\RequirePackage[utf8]{inputenc}
\RequirePackage{amsmath}
\RequirePackage{amsfonts}
\RequirePackage{hyperref}

\newcommand{\dd}{\mathop{\mathrm{d}\!}{}}
\newcommand{\Tr}{\mathop{\mathrm{Tr}\!}{}}
\newcommand{\deriv}[2]{\dfrac{\dd #1}{\dd #2}}
\newcommand{\pderiv}[2]{\dfrac{\partial #1}{\partial #2}}

\bibliographystyle{alpha}

\date{\today}
\author{Bruno Bucciotti}
\title{Quantization of Gauge systems}


\begin{document}
	\maketitle
	
	\begin{abstract}
		\href{https://www.youtube.com/watch?v=IwInqrN_auU}{Trumpets!}
		Seguiremo \cite{dirac} e soprattutto \cite{HT}
		I segni sono un po'a caso quando sono ininfluenti. Sorry..
		
	\end{abstract}

	\tableofcontents
	\clearpage
	\section{Karapet}
	\subsection{Introduzione}
	Molte teorie utilizzano un numero di variabili maggiore del numero di gradi di libertà indipendenti del sistema fisico.
	Queste teorie sono dette di gauge (non è una persona..) e le variabili fisicamente significative sono quelle invarianti sotto le cosiddette \emph{trasformazioni di gauge}. Questa ridondanza permette di rendere più manifeste alcune simmetrie. Tutte le teorie di gauge possono essere scritte come sistemi hamiltoniani vincolati, che procediamo a studiare.
	
	\subsection{Lagrangian formulation}
	Siamo familiari con $L(q,\dot{q})$, $S[q(t)] = \int L(q(t), \dot{q}(t)\, \dd t$, $\delta S=0$. Con i soliti passaggi si arriva alle equazioni di EL
	\[ \deriv{}{t} \left( \pderiv{L}{\dot{q}^n} \right) = \pderiv{L}{q^n} \]
	da cui
	\[ \ddot{q}^{n'} \pderiv{^2 L}{\dot{q}^{n'}\partial\dot{q}^n} = \pderiv{L}{\dot{q}^n} - \dot{q}^{n'} \pderiv{^2 L}{q^{n'}\partial\dot{q}^n} \]
	
	E' chiaro che il problema è ben posto solo se il termine cinetico nella lagrangiana è invertibile, cioè
	\[ \det\left( \pderiv{^2L}{\dot{q}^{n'}\partial\dot{q}^{n}} \right) \]
	Questa è precisamente la proprietà che verrà meno in teorie di gauge.
	
	\subsection{Hamiltonian formulation}
	Vogliamo ora passare alla formulazione hamiltoniana. Definiamo i momenti coniugati
	\[ p_n = \pderiv{L}{\dot{q}^{n}} \]
	
	La non univocità delle $\ddot{q}^n$ si trasforma in
	\[ \delta p_n = \pderiv{p_n}{\dot{q}^{n'}} \delta \dot{q}^{n'} = M_{n n^{'}} \delta \dot{q}^{n'} \]
	dove la matrice $M$ non è invertibile. Dunque variare le velocità fa cambiare gli impulsi, ma in modo non indipendente (non posso ricavare la variazione delle velocità nota la variazione degli impulsi). Questo si traduce nell'esistenza di vincoli (primari) fra i $p_n$
	\[ \phi_m(q,p) = 0,\qquad m=1,\dots,M \]
	di cui $M'$ indipendenti fra loro (questa generalità è necessaria per non dover poi ricavare esplicitamente gli $M'$ vincoli indipendenti).
	
	Una varietà $M'$ dimensionale in $(q,\dot{q})$ viene mappata in un punto $(q,p)$ che soddisfa questi vincoli. Per rendere invertibile la mappa saremo costretti a introdurre $M'$ coordinate per parametrizzare la varietà di partenza.
	
	\paragraph{Esempio}
	Consideriamo la lagrangiana
	\[ L = \dfrac{1}{2} (\dot{q^1}-\dot{q^{2}})^2 \]
	\[ p_1 = \dot{q}^{1} - \dot{q}^{2} \]
	\[ p_2 = \dot{q}^{2} - \dot{q}^{1} \]
	allora
	\[ \phi = p_1 + p_2 = 0 \]
	
	Imponiamo le \emph{regularity conditions} sui $\phi_m$. Deve sempre essere possibile separare le $\phi_m$ in $\phi_{m'},\,m'=1,\dots,M'$ indipendenti e $\phi_{\bar{m'}},\, \bar{m'}=M'+1,\dots,M$ implicate dalle precedenti $M'$, almeno su un insieme di aperti che ricopra la varietà $2N-M'$ dimensionale data da $\phi_m=0$; la separazione può essere diversa su ciascun aperto.
	La richiesta di indipendenza per $\phi_{m'}$ si scrive formalmente come
	\[ \pderiv{\phi_{m'}}{(q,p)} \]
	deve avere rango $M'$. Equivalentemente le funzioni $\phi_{m'}$ devono poter essere prese come completamento regolare del sistema di coordinate sulla varietà $2N-M'$ dimensionale data dai vincoli (in pratica $\phi_{m'}$ parametrizzano come mi allontano dal vincolo).
	
	\paragraph{Esempio} La superficie $p_1=0$ ha lo stesso grafico di $p_1^2=0$ o $\sqrt{p_1}=0$, ma questi due vincoli sono singolari nell'origine, perciò non soddisfano le condizioni di regolarità.
	
	Le condizioni di regolarità assicurano i seguenti due teoremi
	\paragraph{Teorema 1} Qualunque funzione liscia $G$ dello spazio delle fasi che si annulla sul vincolo $\phi_m=0$ è scrivibile come
	$G = g^m \phi_m$ per certe funzioni $g^m(q,p)$.
	\paragraph{Teorema 2} Se $\lambda_n \delta q^n + \mu^n \delta p_n = 0$ per variazioni arbitrarie $\delta q^n$, $\delta p_n$ tangenti al vincolo, allora esistono $u^m(q,p)$ con
	\[ \lambda_n = u^m \pderiv{\phi_m}{q^n},\qquad \mu^n = u^m \pderiv{\phi_m}{p_n} \]
	
	\subsubsection{Hamiltonian}
	Sia ora attrezzati per parlare di Hamiltoniana. Definiamo
	\[ H = \dot{q}^{n}p_n (q,\dot{q}) - L \]
	prendendo la variazione si ha
	\[ \delta H = \dot{q}^{n} \delta p_n + \delta \dot{q}^{n} p_n - \delta \dot{q}^{n} \pderiv{L}{\dot{q}^{n}} - \delta q^n \pderiv{L}{q^n} = \dot{q}^{n} \delta p_n - \delta q^n \pderiv{L}{q^n} \]
	che mostra come $H$ sia funzione di $(q,p)$. Tuttavia la definizione di $H$ contiene una certa arbitrarietà, in quanto potremmo considerare una seconda hamiltoniana uguale alla prima sul vincolo (ma diversa fuori) e ottenere la stessa dinamica fisica. E' chiaro dal teorema 1 che la generica hamiltoniana si può scrivere come
	\[ H \rightarrow H + c^m(q,p) \phi_m \]
	
	Detto altrimenti
	\[ \delta H = \pderiv{H}{q^n} \delta q^n + \pderiv{H}{p_n} \delta p_n \]
	e comparando con la precedente espressione per $\delta H$ e applicando il teorema 2, si ha
	\[ \left( \pderiv{H}{q^n} + \pderiv{L}{q^n} \right) \delta q^n + \left( \pderiv{H}{p_n} - \dot{q}^{n} \right) \delta p_n = 0 \]
	\[ \dot{q}^{n} = \pderiv{H}{p_n} + u^m \pderiv{\phi_m}{p_n} \]
	\[ \dot{p}_n = -\pderiv{H}{q^n} - u^m \pderiv{\phi_m}{q^n} \]
	In conclusione possiamo passare da $(q,\dot{q})$ a $(q,p,u)$+vincolo e viceversa, abbiamo trovato le $M'$ coordinate $u$ che ci mancavano. La trasformazione si scrive
	\[ q^n = q^n, \qquad p_n = \pderiv{L}{\dot{q}^{n}},\qquad u^m = u^m(q,p) \]
	mentre quella inversa è
	\[ q^n = q^n,\qquad \dot{q}^{n} = \pderiv{H}{p_n} + u^m \pderiv{\phi_m}{p_n},\qquad \phi_m(p,q) = 0 \]
	
	\subsubsection{Action principle in hamiltonian form}
	\[ \delta \int_{t_1}^{t_2} \left( \dot{q}^{n} p_n - H - u^m \phi_m \right) \dd t = 0 \]
	con variabili $(q,p,u)$ soggette ai vincoli $\delta q(t_1) = \delta q(t_2)  = 0$. Il principio è invariante per $H\rightarrow H + c^m\phi_m$, che corrisponde a ridefinire le $u^m$. Notare che variare rispetto a $u^m$ impone i vincoli $\phi_m=0$ (moltiplicatori di Lagrange).
	
	Alternativamente possiamo considerare
	\[ \delta \int_{t_1}^{t_2} \left( \dot{q}^{n} p_n - H \right) \dd t = 0 \]
	dove abbiamo risolto i vincoli, cioè imponiamo $\phi_m=0$, $\delta \phi_m = 0$ (gauge unitaria).
	
	L'evoluzione di una generica funzione è
	\[ \dot{F} = [F, H] + u^m[F, \phi_m] \]
	$[F, G]$ il Poisson bracket fra $F$ e $G$.
	
	\subsection{Secondary constraints}
	Certamente dovremo richiedere che
	\[ \deriv{\phi_m}{t} = [\phi_m, H] + u^{m'} [\phi_m, \phi_{m'}] = 0 \]
	che può non essere realizzata trivialmente. In tal caso si ottengono vincoli secondari (anche per loro va richiesta derivata temporale nulla) $\phi_k=0,\,k=M+1,\dots,M+K=J$ (notare che qui sono necessarie le equazioni del moto, mentre nei vincoli primari no). Nel seguito non avremo bisogno di distinguere fra {primary constraints} e {secondary constraints}. Si assumono sempre le condizioni di regolarità.
	
	\subsection{Weak and strong equations}
	Introduciamo il simbolo $\approx$ per indicare che $F\approx G$ se l'uguaglianza è vera restringendosi ai vincoli. Ovviamente quindi $\phi_j=0$.
	
	\subsection{Restrictions on lagrange multipliers}
	Sappiamo che
	\[ [\phi_j, H] + u^m [\phi_j, \phi_m] \approx 0 \]
	dove $m$ è sommato sui soli vincoli primari, $j$ invece sono tutti. Si tratta di $J$ equazioni lineari non omogenee nelle $M$ incognite $u^m$.
	La soluzione generale si scrive come somma di una particolare $U$ e della più generale soluzione dell'omogenea associata $V$.
	\[ u^m = U^m + V^m \]
	l'equazione omogenea è
	\[ V^m[\phi_j, \phi_m] = 0 \]
	e ha $M$ soluzioni $V^m_a$. Concludendo
	\[ u^m = U^m + v^a V_a^m \] 
	dove $v^a$ sono arbitrari.
	
	\subsection{Total hamiltonian}
	\[ \dot{F} = [F, H] + (U^m+v^aV_a^m) [F, \phi_m] \approx [F, H'] + v^a[F, \phi_a] \approx [F, H' + v^a \phi_a] = [F, H_T] \]
	dove si è definito
	\[ H' = H + U^m \phi_m, \qquad \phi_a = V_a^m \phi_m,\qquad H_T = H' + v^a \phi_a \]
	Notiamo che lo split di $H_T$ in $H'$ e $v^a\phi_a$ è arbitrario, difatti potremmo ridefinire la soluzione particolare $U^m$ e i coefficienti arbitrari $v^a$ mantenendo la stessa $H_T$.
	
	\subsection{First and second class constraints}
	Una funzione $F$ si dice \emph{first class} se $[F, \phi_j] \approx 0,\, j=1,\dots,J$. Altrimenti si dice \emph{second class}.
	$\phi_a=V_a^m\phi_m$ sono tutti first class, e anzi sono una base dei first class primary constraints, cioè qualunque first class primary constraint si esprime come combinazione lineare dei $\phi_a$. Anche $H'$ è first class. Il poisson bracket di due funzioni first class è anch'esso first class (identità di Jacobi).
	
	\subsection{Gauge transformation}
	Abbiamo già insistito sull'arbitrarietà delle funzioni $v^a$. Consideriamo due possibili scelte al tempo $t$: $v^a$ e $\tilde{v}^a$. Al tempo $t+\delta t$ la funzione $F$ evolve in
	\[ F(t+\delta t) = F + \delta t [F, H'+v^a\phi_a],\qquad \tilde{F}(t+\delta t) = F + \delta t [F, H'+\tilde{v}^a\phi_a] \]
	Dunque la variazione $\delta F$ è una trasformazione non fisica, con
	\[ \delta F = (\tilde{F}-F)(t+\delta t) = [\delta t (\tilde{v}^a - v^a)] [F, \phi_a] = \delta v^a [F, \phi_a]\]
	In conclusione le trasformazioni generate dai $\phi_a$ non cambiano lo stato fisico del sistema, sono cioè \emph{trasformazioni di gauge}.
	Sono indipendenti fra loro sse i $\phi_a$ sono indipendenti come vincoli.
	
	Postuliamo che tutti i first class constraints generino trasformazioni di gauge, altrimenti non sappiamo come quantizzare la teoria.
	\paragraph{Esempio} teoria $L = \dfrac{1}{2} e^y \dot{x}^2$. TODO
	
	\subsection{Extended hamiltonian}
	Denotiamo $\gamma$ i first class constraints, con $\chi$ i second class. Presi insieme, li indicheremo $\phi_j$.
	Poiché anche i secondary first class constraints generano trasformazioni di gauge, dobbiamo definire l'\emph{extended hamiltonian}
	\[ H_E = H' + u^a \gamma_a \]
	L'evoluzione prevista da $H',H_T,H_E$ è la stessa sul vincolo, mentre differisce fuori. Possiamo dedurre la dinamica da un principio di minimo utilizzando come azione
	\[ S_E = \int (\dot{q}^n p_n - H' - u^j \phi_j) \dd t \]
	dove ricordiamo che $\gamma^a$ si annullano sul vincolo, perciò si scrivono come combinazione lineare dei $\phi_j$, e dove abbiamo definito $u^j$ come ($u^a$ arbitrarie)
	\[ \gamma_a = A_a^j \phi_j,\qquad u^j = A_a^j u^a \]
	
	La dinamica si scrive esplicitamente come
	\[ \dot{F} \approx [F, H_E],\qquad \phi_j \approx 0 \]
	
	\subsection{Examples}
	\subsubsection{Example 1}
	\[ L = \dfrac{1}{2} \sum_{i=1}^{n-1} (q^i-\dot{q}^{i+1})^2 \]
	calcoliamo gli impulsi
	\[ \pi_1 = 0,\qquad \pi_i = \dot{q}^i-q^{i-1},\,i=2,\dots,n \]
	e l'hamiltoniana
	\[ H = \dfrac{1}{2} \sum_{i=2}^{n} \left(\pi_i^2 + \pi_i q^{i-1}\right) \]
	C'è un solo vincolo primario $\phi_1 = \pi_1$, dunque l'hamiltoniana totale è
	\[ H_T = \dfrac{1}{2} \sum_{i=2}^{n} \left(\pi_i^2 + \pi_i q^{i-1}\right) + u \pi_1 \]
	e con essa possiamo ricavare i vincoli secondari. Imponendo $\dot{\pi_1}\approx 0$ si ricava $\pi_2 \approx 0$. Imponendo $\dot{\pi_2} \approx 0$ \dots cioè $\pi_i \approx 0\,\forall\,i$. Banalmente
	\[ [\pi_i, \pi_j] \approx 0 \]
	da cui si tratta di tutti vincoli di prima classe. Non ci sono gradi di libertà fisici.
	L'hamiltoniana estesa è
	\[ H_E = H + u^i \pi_i \]
	
	Si verifica che la seguente è una trasformazione di gauge
	\[ \delta q^i = \deriv{^{n-i} \epsilon}{t^{n-i}} \]
	e che è generata da $G = \sum \deriv{^{n-i} \epsilon}{t^{n-i}} \pi_i$. La regola generale per costruire $G$ a partire dai vincoli sarà data in seguito.
	
	\subsubsection{Example 2}
	\[ q^i,\,i=1,\dots,n,\qquad L=0 \]
	Chiaramente $\pi_i = 0$ primary first class constraints, $H=0$, $H_E = u^i \pi_i$. Le trasformazioni di gauge sono generate da $G = \sum \epsilon^i \pi_i$, $\epsilon^i$ libere. Tutti i gradi di libertà sono gauge, quindi il sistema è lo stesso di prima. Il caso precedente è riconducibile a questo mediante parziale gauge-fixing
	\[ \epsilon^i = \deriv{\epsilon^{i+1}}{t} \]
	
	Notare che nei due esempi i vincoli compaiono in modo diverso (primary e secondary) nella formulazione lagrangiana, mentre in quella puramente hamiltoniana la distinzione si perde.
	
	\subsubsection{Example 3}
	\[ L = -\dfrac{1}{2} \sum (q^i)^2 \]
	si ricavano i constraint primari e l'hamiltoniana
	\[ \pi_i = 0,\qquad H = \dfrac{1}{2} \sum (q^i)^2 \]
	da cui
	\[ H_T = \dfrac{1}{2} \sum (q^i)^2 + \sum u^m \pi_m \]
	e infine i secondary constraints sono
	\[ \dot{\pi}_i = [\pi_i, H_T] = -q^i \approx 0 \]
	
	Possiamo vedere anche questo come lo stesso sistema dei due esempi precedenti, dove abbiamo fissato la gauge data dai $\pi_i$ mediante i secondary constraints. Tutti i vincoli sono di seconda classe.
	
	Abbiamo delle restrizioni sui moltiplicatori di lagrange $u^m$. Difatti mentre $\dot{\pi}_m\approx 0$ dà i vincoli secondari,
	\[ \dot{q}^m = [q^m, H]+ u^{m'}[q^m, \pi_{m'}] = u^{m'} \delta_{m'}^m = u^m \approx 0\]
	Dunque $u^m \approx 0$. Notare che il sistema di equazioni si riduce in questo caso a uno omogeneo ($[q^m, H]=0$); inoltre il sistema è di $n$ equazioni ($\forall\,q^m$) in $n$ incognite ($u^m$), dunque esiste solo la soluzione banale.
	% DOmandare a Karapet per le varie hamiltoniane. A me sembra che i vincoli secondari si trovino con H_T, dove u^m vanno arbitrariamente spezzati in U^m e V^m. Ma se lo split è arbitrario, l'extended hamiltonian è un po'maldefinita. Inoltre nell'esempio 3 qui sopra io direi che la soluzione particolare è u=0, quindi H' = H, non ci sono first class constraints, perciò H_E = H. Tuttavia a lezione abbiamo detto diverso.
	
	\subsubsection{Example: Electromagnetism}
	\[ L = \int \dd^3 x\, -\dfrac{1}{4} F_{\mu\nu}F^{\mu\nu},\qquad F_{\mu\nu} = \partial_\mu A_\nu - \partial_\nu A_\mu \]
	dove $A_\mu$ è l'unica variabile, $F_{\mu\nu}$ è solo un placeholder. Calcoliamo gli impulsi coniugati a $A_\mu$. Metrica \emph{mostly plus}, $a$ indici spaziali.
	\[ \pi^\mu (x) = \pderiv{L}{(\partial_0 A_\mu)} = -\dfrac{1}{2} F^{\rho\sigma} \pderiv{F_{\rho\sigma}}{(\partial_0 A_\mu)} = -\dfrac{1}{2} F^{\rho\sigma} (\delta^0_\rho\delta^\mu_\sigma - \delta^0_\sigma\delta^\mu_\rho) = F^{\mu 0} \]
	Notiamo che $\pi^0=0$ è un vincolo primario poiché $F^{00}=0$, e che $\pi^a = F^{a0} = \partial_0 A_a - \partial_a A_0$. Calcoliamo l'hamiltoniana
	\[ H = \int F^{a0} \partial_0 A_a + \dfrac{1}{2} F^{a0}F_{a0} + \dfrac{1}{4} F^{ab}F_{ab}\, \dd^3 x \]
	\[ = \int \pi^a(\pi^a + \partial_a A_0) -\dfrac{1}{2} \pi^a \pi^a + \dfrac{1}{4} F^{ab} F_{ab}\, \dd^3 x \]
	\[ = \int \mathcal{H}\, \dd^3 x,\qquad \mathcal{H} = \dfrac{1}{2} \pi^a \pi^a +\dfrac{1}{4} F^{ab}F_{ab} - A_0 \partial_a \pi^a \]
	dove nell'ultimo passaggio abbiamo integrato per parti. Calcoliamo ora il vincolo secondario associato a $\pi^0$, 
	ricordando che $[A_\mu(x), \pi_\nu (y)] = \eta_{\mu\nu} \delta^3(x-y)$
	\[ \dot{\pi}^0(x) = [\pi^0(x), H] = \int \dd^3 y\, [\pi^0(x), \mathcal{H}(y)] = \int \dd^3 y\, \delta^0_0 \delta^3(x-y) \partial_a \pi^a = \partial_a \pi^a \approx 0 \]
	Quest'ultima è la prima legge di Maxwell.
	%ATtenzione: Karapet dice che [\pi, A] = +, mentre io avrei detto il contrario (e qui sopra ho fatto il contrario). Domandare se typo o cosa.
	Continuiamo a cercare secondary constraints. E' più comodo studiare $f=\int \lambda(x) \partial_a \pi^a(x)\, \dd^3 x$, che si riscrive
	\[ f=-\int \pi^a \partial_a \lambda\, \dd^3 x \]
	Otteniamo quindi
	\[ \dot{f} = [f, H] + u[f, \pi^0] = -\int \lambda \partial_a\partial_b F^{ab}\, \dd^3 x = 0 \]
	Abbiamo dunque esaurito i vincoli.
	\[ H_E = \int \left( \dfrac{1}{2} \pi^a\pi^a + \dfrac{1}{4} F^{ab}F_{ab} \right) + (u-A_0) \partial_a \pi^a + v \pi^0\, \dd^3 x \]
	dove ora rinominiamo $u' = u-A_0$. Si trova che
	\[ \dot{A}_0 = [A_0, H_E] = v \]
	Possiamo in realtà risolvere il vincolo $\pi^0 = 0$ ponendo $A_0 = \pi^0 = 0$ in tutte le nostre formule, ottenendo quindi
	\[ S_E = \int \pi^a \dot{A}_a - \dfrac{1}{2} \pi^a\pi^a -\dfrac{1}{4} F^{ab}F_{ab} + u \partial_a \pi^a\, \dd^3x \]
	
	\subsection{Second class constraints}
	Siano $\phi_j$ tutti i vincoli. Definiamo la matrice antisimmetrica $C_{jj'}$ come
	\[ C_{jj'} = [\phi_j, \phi_{j'}] \]
	e $C_{jj'} \approx 0$ \emph{sse} non ci sono second class constraints (segue dalla definizione). Inoltre $\det C_{jj'} \approx 0$ \emph{sse} esiste almeno un first class constraint. Dimostrazione: la freccia inversa è banale, quella diretta si fa considerando $\lambda_j\neq 0$ t.c. $\lambda_j C_{jj'} = 0$ (esiste perché $\det=0$); allora $\lambda_j \phi_j$ è first class, difatti
	\[ [\lambda_j \phi_j, \phi_{j'}] \approx \lambda_j [\phi_j, \phi_{j'}] = \lambda_j C_{jj'} \approx 0 \]
	
	Possiamo sempre portare $C_{jj'}$ nella forma
	\[ C_{jj'} = \begin{pmatrix}
	0 & 0 \\
	0 & C_{\alpha\beta} \\
	\end{pmatrix} \]
	a patto di andare nella base in cui prima elenchiamo i first class constraints, d'ora in poi $\gamma_a$, e poi i second class $\chi_a$.
	Notiamo che $C_{\alpha\beta}$ è una matrice antisimmetrica con determinante non nullo (se avesse $\det=0$ potrei azzerare una colonna con un cambio base, ma per antisimmetria avrei ridotto le dimensioni di $C_{\alpha\beta}$); ogni first class constraint è combinazione dei $\gamma$, mentre ogni second class deve essere combinazione di \emph{almeno un} $\chi$.
	
	Notiamo infine che lo split è preservato dai cambi base che rimescolano le $\gamma$ fra loro senza far perdere informazione, cioè
	\[ \gamma_a \rightarrow A_a^b \gamma_b + T^{bc}_a \chi_b\chi_c,\qquad \chi_a = B_a^b \gamma_b + C_a^b \chi_b \]
	\[ \det A \neq 0,\qquad \det C \neq 0 \]
	Il numero di $\chi_a$ è pari per dof bosonici. Richiediamo che $\det C_{\alpha\beta} \neq 0$ su tutto $\chi_a\approx 0$, anche quando $\gamma_a \approx 0$ non è soddisfatta.
	
	\subsubsection{Dirac bracket}
	Vogliamo capire come gestire i second class constraints, visto che generano trasformazioni che, per definizione, in generale mappano uno stato fisico in uno non fisico.
	\paragraph{Esempio}
	Supponiamo $n$ variabili $q^i$, $p_i$, con i vincoli $q^1 \approx 0$, $p_1 \approx 0$ (second class).
	
	Possiamo definire il Dirac bracket in modo analogo al Poisson bracket, ma escludendo il primo addendo dalla somma
	\[ [F,G]^* = \sum_{i=2}^{n} (\dots) \]
	Abbiamo cioè posto $\chi_a=0$ \emph{strongly}, osservando inoltre che le equazioni del moto per i restanti gradi di libertà non cambiano. Il bracket di Dirac soddisfa tutte le proprietà del bracket di Poisson (ovviamente, visto che possiamo considerare il sistema con $n\ge 2$).
	
	\vspace{5mm}
	La soluzione generale è dovuta a Dirac e procede così: definiamo $C^{\alpha\beta}$ come l'inverso di $C_{\alpha\beta}$ e il bracket come
	\[ [F,G]^* = [F, G] - [F, \chi_\alpha] C^{\alpha\beta} [\chi_\beta, G] \]
	Dimostrazione delle proprietà del bracket di Dirac:
	
	per prima cosa osserviamo che data $F$ possiamo sempre costruire $F^*=F+v^\beta\chi_\beta$ con $[F^*, \chi_\alpha] \approx 0$. Difatti
	\[ [F^*, \chi_\alpha] \approx [F, \chi_\alpha] + v^\beta [\chi_\beta, \chi_\alpha] \approx 0 \]
	da cui
	\[ v^\beta C_{\beta\alpha} \approx - [F, \chi_\alpha] \]
	Invertendo $C$ si ricava
	\[ v^\beta C_{\beta\alpha} C^{\alpha\gamma} = v^\gamma = -[F, \chi_\alpha] C^{\alpha\gamma}\rightarrow v^\beta = -C^{\alpha\beta} [F, \chi_\alpha] \]
	da cui $[F^*, G]$ è una riscrittura debole del bracket di Dirac
	\[ [F^*, G] \approx [F, G] + v^\beta [\chi_\beta, G] = [F, G] - [F,\chi_\alpha] C^{\alpha\beta} [\chi_\beta, G] \]
	e infine, usando $[\chi_\beta, G^*] \approx 0$, dimostriamo che starrare $G$ non ha effetti se $F$ è già starrato
	\[ [F^*, G^*] \approx [F, G^*] \approx [F^*, G] \]

	Con queste due ultime identità diventa un facile esercizio dimostrare che il bracket di Dirac soddisfa tutte le proprietà richieste, che elenchiamo
	\[ [F,G]^* = -[G, F]^* \]
	\[ [F, GR]^* = [F,G]^* R + G[F,R]^* \]
	\[ [[F,G]^*, R]^* + cyc = 0 \]
	\[ [\chi_\alpha, F]^* = 0,\quad \forall\,F \]
	\[ [F, G]^* \approx [F, G],\quad \mathrm{G\, first\, class,\, F\, arbitrary} \]
	\[ [R, [F, G]^*]^* \approx [R, [F, G]^*]^*,\quad \mathrm{F,G\, first\, class,\, R\, arbitrary} \]
	
	Osserviamo che $H_E$, essendo first class, genera la stessa dinamica con entrambi i bracket.
	Lavorando con il bracket di Dirac, i second class constraints diventano \emph{identità}.
	
	\subsubsection{Example: Proca theory}
	\[ L = \int \dd^3 x\, \left(-\dfrac{1}{4} F_{\mu\nu}F^{\mu\nu} - \dfrac{1}{2} m^2 A_\mu A^\mu \right) \]
	ricaviamo
	\[ \pi^\mu = F^{\mu 0} \]
	cioè il vincolo primario $\pi^0=0$ e $\pi^a = F^{a0}$. Ricavando la densità hamiltoniana si ha
	\[ \mathcal{H} = \dfrac{1}{4} F^{ab}F_{ab} + \dfrac{1}{2} \pi^a\pi^a + \dfrac{1}{2} m^2 A_a^2 - \dfrac{1}{2} A_0^2 - A_0 \partial_a \pi^a \]
	\[ H_T = H + \int u\pi^0\,\dd^3 x \]
	Vincoli secondari
	\[ \dot{\pi}^0 \approx \left[ \pi^0, \int -\dfrac{1}{2} m^2 A_0^2 - A_0 \partial_a \pi^a\,\dd^3 y \right] \approx m^2 A_0 + \partial_a \pi^a \approx 0 \]
	Posto ora $f = \int \lambda \left( m^2 A_0 + \partial_a \pi^a \right)$ si ricava
	\[ \dot{f} \approx [f,H] + \int u [f,\pi^0]\,\dd^3 x \approx m^2 \int \lambda \left(u-\partial_a\pi^a\right)\,\dd^3 x \]
	Abbiamo perciò una condizione sul lagrange multiplier $u$, nessun nuovo vincolo secondario.
	
	Riassumendo i vincoli sono $\pi^0 = 0$, $m^2A_0 + \partial_a \pi^a = 0$. Il loro bracket si calcola facilmente
	\[ [\phi_1(x), \phi_2(y)] = -m^2 \delta^3(x-y) \]
	%E-LA-DELTA-CHE-FA-DORME?
	\[ C_{\alpha\beta} = \begin{pmatrix}
	0 & -m^2 \\
	m^2 & 0 \\
	\end{pmatrix},\qquad
	C^{\alpha\beta} = \begin{pmatrix}
	0 & \dfrac{1}{m^2} \\
	-\dfrac{1}{m^2} & 0 \\
	\end{pmatrix} \]
	
	Il bracket di Dirac è dato dalla formula, ed è
	\[ [F,G]^* = [F,G] - [F,\phi_1]\dfrac{1}{m^2}[\phi_2,G] + [F,\phi_2]\dfrac{1}{m^2}[\phi_1,G] \]
	L'hamiltoniana si può riscrivere con i vincoli risolti come identità
	\[ H = \int \dfrac{1}{4} F^{ab}F_{ab} + \dfrac{1}{2} \pi^a\pi^a + \dfrac{1}{2} m^2 A_a^2 + \dfrac{1}{2m^2} (\partial_a \pi^a)^2\,\dd^3 x \]
	dove le uniche variabili ora sono $A_a,\pi^a$.
	
	\subsection{Reducible first class and second class}
	TODO. Vedi 1.3.4 di \cite{HT}.
	
	\subsection{More on Maxwell theory}
	L'elettrodinamica presenta la simmetria di gauge $A_\mu\rightarrow A_\mu + \partial_\mu \sigma$. La formulazione hamiltoniana tuttavia è al \emph{primo ordine}, con tempo e spazio separati. In questo linguaggio quindi la simmetria di gauge prende una forma diversa.
	
	Ricordiamo che ci sono due vincoli (first class): $\pi^0\approx 0$ e $\partial_a \pi^a \approx 0$. Calcoliamo la variazione di gauge indotta da ciascuno. Introduciamo la funzione di test $\lambda(x)$ e $F$ qualunque e abbiamo
	\[ \delta F = \epsilon \left[F, \int \lambda(y) \pi^0(y)\,\dd^3y\right] = \epsilon \int \lambda(y) \dfrac{\delta F}{\delta A_0(y)}\,\dd^3 y ,\quad \delta A_0(x) = \epsilon \lambda(x),\quad \delta A_a = 0\]
	Deduciamo che $\pi^0(x)$ genera la trasformazione di gauge infinitesima $A_0(x)\rightarrow A_0(x) + \phi(x)$. Analogamente, e integrando per parti,
	\[ \delta F = \epsilon \left[ F, \int \lambda(y) \partial_a \pi^a(y)\, \dd^3 y \right] = -\epsilon \int \partial_a \lambda (y) \dfrac{\delta F}{\delta A_a(y)}\, \dd^3 y,\quad \delta A_0 = 0,\quad \delta A_a(x) = - \epsilon \partial_a \lambda(x) \]
	dunque $\partial_a \pi^a$ genera le trasformazioni $A_a\rightarrow A_a + \partial_a \sigma$.
	Possiamo quindi identificare $\phi = \partial_0 \sigma$.
	
	\subsection{Gauge fixation}
	Come abbiamo visto i first class constraints generano trasformazioni di gauge, cioè mappano punti dello spazio delle fasi $(q,p)$ in altri punti $(q',p')$ corrispondenti allo stesso stato fisico. Si parla quindi di \emph{gauge orbit} per indicare l'insieme dei punti corrispondenti allo stesso stato (è un'orbita sotto l'azione del gruppo di gauge).
	
	Il nostro desiderio è introdurre a mano dei vincoli, non presenti di per sè nella teoria, che selezionino un punto, un rappresentante, da ogni orbita. Si parla di \emph{gauge fixing}, in quanto l'invarianza di gauge è rotta e alle variabili non fisiche è assegnato un preciso valore. Chiamiamo questi vincoli $C_b(q,p)\approx0$.
	
	Imponiamo due condizioni su questi vincoli
	\begin{itemize}
		\item Accessibilità: dato un punto $(q,p)$ deve esistere una trasformazione di gauge che mappi $(q,p)\rightarrow(q',p')$ soddisfacente $C_b\approx 0$.
		\item Complete fixing: l'unica trasformazione di gauge che preservi $C_b\approx 0$ deve essere l'identità.
	\end{itemize}
	Le due condizioni insieme implicano che il numero di condizioni di gauge indipendenti deve essere uguale al numero di first class constraints indipendenti. In particolare il complete fixing si scrive come
	\[ \delta u^a [C_b, \gamma_a] \approx 0 \rightarrow u^a = 0 \]
	che, oltre che implicare da solo che $b\ge a$, se unito alla condizione di accessibilità implica che la matrice \emph{quadrata} $[C_b,\gamma_a]$ è invertibile: $\det [C_b,\gamma_a] \neq 0$.
	
	Come abbiamo visto, questa condizione ci dice che il set $\{\gamma_a,C_b\}$ è un insieme di second class constraints. Concludiamo quindi che non ci sono first class constraints rimasti, ma come potrebbero? Se ci fossero, genererebbero trasformazioni di gauge, e noi abbiamo chiesto complete gauge fixing.
	
	\subsubsection{Counting of degrees of freedom}
	Sia $N_\gamma$ il numero di (ex) first class constraints, che abbiamo visto essere uguale al numero di $\{C_b\}$ (indipendenti). Sia $N_\chi$ il numero di second class constraints $\chi$. Sia $N$ il numero di gradi di libertà fisici. Sia $N_T$ il numero di variabili canoniche totali.
	
	Per ogni grado di libertà fisico ci sono due phase space variables indipendenti. Queste si enumerano contando quelle totali e sottraendo il numero di vincoli, che hanno origine dagli (ex) first class constraints, dal gauge fixing e dai second class constraints. In formule
	\[ 2N = N_T - N_\gamma- N_\gamma - N_\chi \rightarrow N = \dfrac{1}{2} (N_T-N_\chi-2N_\gamma) \]
	
	Il conteggio è ben definito per $N$ finito e porta sempre a $N$ intero ($N_T,N_\chi$ pari, almeno per dof bosonici).
	
	\subsection{Continuum limit caveat}
	Nel caso continuo, che abbiamo già affrontato in modo "informale", abbiamo scritto spesso $\int u^a(x) \gamma_a(x)\,\dd^3 x$. A quale classe appartiene $u^a(x)$? Dobbiamo certamente garantire che
	\[ \delta F = \int u^a(x) [F,\gamma_a(x)]\,\dd^3x \]
	sia una trasformazione di gauge.
	
	Non esiste risposta generale a questa domanda, che è anzi oggetto di studio. Ci limitiamo a menzionare alcune delle sottigliezze fisicamente interessanti che possono emergere. Per campi di gauge che si annullano a infinito, $u$ che tendono a un valore costante sono legati ad una carica totale: \href{https://arxiv.org/pdf/1803.10194.pdf}{Asymptotic symmetries of electromagnetism at spatial infinity}.
	Possono esistere trasformazioni non connesse con l'identità (non scrivibili come trasformazioni infinitesime iterate): large-gauge transformations. Infine GR in 4d ha un gruppo di gauge infinito dimensionale asintotico.
	
	\subsubsection{Do all second class constraints arise from gauge fixation?}
	I first class constraints diventano second class dopo il gauge fixing. E' sempre possibile vedere i second class constraints come first class dopo un opportuno gauge fixing? Sì, in modo non univoco. Facciamo due esempi.
	\paragraph{Example 1} Consideriamo un sistema con un solo grado di libertà e con $q^1=p_1=0$ second class. Possiamo vedere $q^1=0$ come gauge fixing della simmetria data da $p_1\approx 0$, cioè $q^1\rightarrow q^1 + \epsilon$. In questa teoria si ha quindi solo $p_1\approx 0$, first class.
	
	In questo caso abbiamo cestinato alcuni vincoli (che abbiamo reinterpretato come gauge fixing), mentre i restanti erano first class. In generale non è possibile selezionare alcuni vincoli second class da cestinare, serve introdurre nuove variabili.
	\paragraph{Example 2} Stesso sistema di prima, ma possiamo invece introdurre $q^2,p_2$ con
	\[ \phi_1 = q^1+q^2 = 0 \]
	\[ \phi_2 = p_1-p_2 = 0 \]
	Questi vincoli sono first class e generano l'invarianza di gauge
	\[ q^1 \rightarrow q^1 + \epsilon_2,\qquad p_1 \rightarrow p_1 - \epsilon_1 \]
	\[ q^2 \rightarrow q^2 - \epsilon_2,\qquad p_2 \rightarrow p_2 - \epsilon_1 \]
	Imporre $q^2=p_2=0$ ci riporta al sistema precedente.
	
	Sembra che questo secondo esempio sia inutile, poiché il primo esempio era lo stesso sistema fisico, ma più semplice. In realtà in questo caso abbiamo barattato due second class constraints con due first class, al prezzo di nuove variabili di gauge. Il vantaggio è il non dover introdurre il Dirac bracket, e questo rende possibile quantizzare la teoria nel modo usuale.
	
	La conclusione è che è sempre possibile sia fare di disfare il gauge fixing (almeno localmente, ad esempio nel caso di gauge fixing globale ci possono essere ostruzioni topologiche, dette di Gribov).
	
	
	
	\newpage
	\bibliography{bibliografia}
\end{document}